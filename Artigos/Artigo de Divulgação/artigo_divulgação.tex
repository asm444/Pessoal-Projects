\documentclass[12pt,a4paper,oneside,brazil]{abntex2}
\usepackage[brazil]{babel}
\usepackage[utf8]{inputenc}
\usepackage[osf]{mathpazo}
\renewcommand{\familydefault}{\rmdefault}
\pagestyle{headings}
\setcounter{secnumdepth}{3}
\usepackage{lmodern}
\setcounter{tocdepth}{3}
\usepackage{amsmath, amsfonts, amssymb, amsthm, array}
\usepackage{float} 
\usepackage{calc, cases}
\usepackage[makeroom]{cancel}
\usepackage{nicefrac}

\usepackage{booktabs} 
\usepackage{subfig}
\usepackage{latexsym}
\usepackage[active]{srcltx}
\usepackage{graphicx} 
\graphicspath{ {figures/} }
\usepackage{indentfirst}
\usepackage{xcolor} 
\usepackage{url} 
\usepackage{relsize} 
\usepackage{microtype}
\OnehalfSpacing 
\makeatletter 
\pdfpageheight\paperheight 
\pdfpagewidth\paperwidth 
\providecommand{\tabularnewline}{\\} 
\raggedbottom 
\bookmarksetup{numbered} 
\setlength{\parindent}{1.3cm} 
\renewcommand{\cftsubsectionfont}{\footnotesize\normalfont\rmfamily} 
\renewcommand{\cftsectionfont}{\normalfont\rmfamily} 
\renewcommand{\ABNTEXchapterfont}{\rmfamily\fontseries{b}\selectfont} 
\usepackage{braket}
\usepackage{tikz}
\usepackage{tikz-3dplot}
\usetikzlibrary{hobby}
\usetikzlibrary{angles}
\usetikzlibrary{quotes}
\usepackage{eufrak}
\usepackage{epstopdf}
\usepackage[makeroom]{cancel}
\usepackage{physics}

%% Índices

\theoremstyle{definition}
\newtheorem{defin}{Definição}
\numberwithin{defin}{section}

\newtheorem{thm}{Teorema}
\numberwithin{thm}{section}

\newtheorem{notation}{Notação}
\numberwithin{notation}{section}

\theoremstyle{remark}

\newtheorem*{sol}{Solução}

\newtheorem{exmp}{Exemplo}
\numberwithin{exmp}{section}

\newtheorem*{obs}{Observação}
\newtheorem*{obss}{Observações}

\newtheorem*{prop}{Propriedade}
\newtheorem*{props}{Propriedade}

\newtheorem{p}{Proposição}
\numberwithin{p}{section}

\newtheorem{lema}{Lema}
\numberwithin{lema}{section}


%% Exemplos Especiais

\newtheorem*{exmp1}{Exemplo (Comutador)}


\newcommand\restr[2]{{
		\left.\kern-\nulldelimiterspace 
		#1 
		\vphantom{\big|} 
		\right|_{#2} 
}}

\newcommand{\xRightarrow}[2][]{\ext@arrow 0359\Rightarrowfill@{#1}{#2}}

\newcommand\numeq[1]%
{\stackrel{\scriptscriptstyle(\mkern-1.5mu#1\mkern-1.5mu)}{=}}

% Esse pacote pode ser editado. Por exemplo, onde está escrito {Autor}, dentro dos colchetes você pode colocar o seu nome, e assim por diante.

\hypersetup{
	pdftitle={Relatório de Iniciação Científica}, 
	pdfauthor={ARTHUR DE SOUZA MOLINA},
	pdfsubject={\imprimirpreambulo},
	pdfcreator={abnTeX2},
	pdfkeywords={Ferramentas para Explorar
		a Radiação Cósmica de Fundo e suas Anomalias}{Cosmologia}{Cálculo Numérico}{Estatística}{Correlação Cruzada},
	colorlinks=true,
	linkcolor=black, %Você pode mudar os cores dos links, i.e red, green, blue, etc.
	citecolor=black,
	filecolor=black,
	urlcolor=black,
	bookmarksdepth=4
}

\makeatother 



%% Novo estilo
\makepagestyle{estilo_pretextual} %%% escolha um nome
\makeevenhead{estilo_pretextual}{}{}{\ABNTEXfontereduzida \textbf \thepage}
\makeoddhead{estilo_pretextual}{}{}{\ABNTEXfontereduzida \textbf \thepage}

%% Customiza comando \pretextual
\renewcommand{\pretextual}{
	\pagenumbering{roman} %%% ou \pagenumbering{Roman}
	\aliaspagestyle{chapter}{estilo_pretextual}
	\pagestyle{estilo_pretextual}
	\aliaspagestyle{cleared}{empty}
	\aliaspagestyle{part}{estilo_pretextual}
}

\begin{document}
	
	\bookmarksetupnext{rellevel=-1}
	
	\pdfbookmark[1]{Capa}{Capa}
	\thispagestyle{empty}
	\begin{center}
		\includegraphics[width=14.8cm]{figuras/logo} 
		\par\end{center}
	
	\begin{center}
		{\color{green!45!black} \rule{1\columnwidth}{1.5mm}}
		\par\end{center}
	
	\begin{center}
		\medskip{}
		\par\end{center}
	
	\begin{center}
		{\Large{}ARTHUR DE SOUZA MOLINA}
		\par\end{center}{\Large \par}
	
	\begin{center}
		\vfill{}
		\par\end{center}
	
	\begin{DoubleSpace}
		\begin{center}
			\textbf{\Large{}Termodinâmica de Buracos Negros AdS}
			\par\end{center}{\Large \par}
	\end{DoubleSpace}
	
	\begin{center}
		\vfill{}
		\par\end{center}
	
	\begin{center}
		{\color{green!45!black} \rule{1\columnwidth}{1.5mm}}
		\par\end{center}
	
	\begin{center}
		Londrina \\
		2022
		\par\end{center}
	
	\cleardoublepage{}
	
	\pdfbookmark[1]{Folha de rosto}{Folha de rosto}
	\thispagestyle{empty}
	\begin{center}
		{\Large{}ARTHUR DE SOUZA MOLINA}
		\par\end{center}{\Large \par}
	
	\begin{center}
		\vfill{}
		\par\end{center}
	
	\begin{DoubleSpace}
		\begin{center}
			\textbf{\Large{}Termodinâmica de Buracos Negros AdS}
			\par\end{center}{\Large \par}
	\end{DoubleSpace}
	
	\begin{center}
		\vfill{}
	
		\par\end{center}
	
	\noindent \begin{flushright}
		\begin{minipage}[c]{9.5cm}%
			Artigo de Divulgação Científica apresentado ao Departamento de Física da Universidade Estadual de Londrina, como requisito para a disciplina Introdução à Técnicas de Ensino e Pesquisa em Física.
		\end{minipage}
		\par\end{flushright}
	
	\vfill{}
	
	\begin{center}
		Londrina\\
		2021
		\par\end{center}
	
	\newpage{}
	
	
		
	\setlength{\absparsep}{18pt}
	
	\begin{resumo}
		
		teste
		
		\textbf{Palavras-chave:} Termodinâmica. Buracos Negros. Radiação Hawking. Espaço AdS. Constante Cosmológica.
	\end{resumo}
	
	
	
	\begin{SingleSpace}
		\noindent Molina, Arthur de Souza \textbf{Tools for Exploring Cosmic Background Radiation and its Anomalies}. 2021.
		Scientific Initiation Report in Physics \textendash{} Universidade
		Estadual de Londrina, Londrina, 2022.
	\end{SingleSpace}
	
	\setlength{\absparsep}{18pt}
	
	\begin{resumo}[Abstract]
		
		\begin{otherlanguage*}{english}
			test 
			
			\textbf{Keywords:} Cosmology. Numerical Calculation. Statistics. Cross-correlation.
		\end{otherlanguage*}
		
	\end{resumo}
	
	\pdfbookmark[0]{\listfigurename}{lof}
	
	\listoffigures*
	
	
	%\pdfbookmark[0]{\listtablename}{lot}
	
	%\listoftables*
	
	
	\cleardoublepage
	
	\pdfbookmark[0]{\contentsname}{toc}
	
	\tableofcontents*
	
	\cleardoublepage
	
	\textual
	\pagenumbering{arabic}
	\setcounter{page}{1}

	
	\postextual
	\bibliographystyle{abntex2-num}
	
	\include{Textos/Introducao.tex}
	
\end{document}