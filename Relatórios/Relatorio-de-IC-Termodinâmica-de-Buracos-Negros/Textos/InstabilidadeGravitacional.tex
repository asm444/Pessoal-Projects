	\chapter*{Instabilidade Gravitacional}
\addcontentsline{toc}{chapter}{Instabilidade Gravitacional}

Quando estudamos uma estrutura de grande escala utilizando as teorias Newtonianas, pressupomos que a estrutura não possui velocidade relativística, a estrutura está em uma região que se expande menos rapidamente em relação as outras regiões, a estrutura esteja em colapso sobre a sua auto-gravidade, ou seja, a instabilidade gravitacional é o mecanismo de maior contribuição para sua evolução e a matéria se comporta como um fluído perfeito.

A partir dessas pressuposições, podemos descrever essas estruturas em termos de sua distribuição de energia $\varepsilon(\mathbf{x},t)$, a entropia por unidade de massa $\textbf{S}(\mathbf{x},t)$ e o vetor velocidade $\textbf{V}(\mathbf{x},t)$, ao estabelecer essas quantidades para um volume fixo em uma região, sabemos que a variação da massa deve ser equivalente a variação de sua distribuição de energia em todo o volume nesta região, em outras palavras, 

\begin{equation}\label{eq1}
	\frac{dM}{dt} = \int_{\Delta V} \frac{\partial \varepsilon(\mathbf{x},t)}{\partial t} dV.
\end{equation}

A variação da massa também pode ser escrita como a variação do fluxo da distribuição de energia em todo o contorno do volume fixo

\begin{equation}\label{eq2}
	\frac{dM}{dt} = - \oint \varepsilon(\mathbf{x},t)\mathbf{V} d\sigma = - \int_{\Delta V} \nabla (\varepsilon(\mathbf{x},t)\mathbf{V}) dV.
\end{equation}

Devido a equivalência de eq. (1) e eq. (2), podemos escrever a seguinte relação consistente

\begin{equation}\label{eq3}
	\frac{\partial \varepsilon}{\partial t} + \nabla (\varepsilon\mathbf{V}) = 0,
\end{equation}

que nos permite lidar com a massa em termos de sua distribuição de energia e volume explicitamente. Podemos descrever instabilidade gravitacional na região, utilizando o conceitos como potencial gravitacional $\phi$, a segunda Lei de Newton e a pressão no fluído. A força gravitacional pode ser escrita como

\begin{equation}\label{eq4}
	\textbf{F}_{gr} = - \Delta M \nabla\phi,
\end{equation}

a força devido a pressão no fluído é dada por 

\begin{equation}\label{eq5}
	\textbf{F}_{pr} = - \oint p \cdot d\sigma = - \int_{\Delta V} \nabla p\,\,\, dV,
\end{equation}

ao utilizar a segunda Lei de Newton, podemos encontrar a equação de Euler

\begin{equation}\label{eq6}
	\dfrac{\partial \textbf{V}}{\partial t} + (\textbf{V} \cdot \nabla) \textbf{V} + \dfrac{\nabla p}{\varepsilon} + \nabla\phi = 0
\end{equation}

 que nos permite descrever a instabilidade gravitacional em termos do volume, distribuição de energia, a pressão e potencial gravitacional.
 
 A conservação de entropia do sistema não permite a dissipação de energia, e portanto, a entropia para um pequeno elemento de matéria é conservada
 
 \begin{equation}\label{eq7}
 	\dfrac{d S(\textbf{x},t)}{dt} = \dfrac{\partial S}{\partial t} + (\textbf{V} \cdot \nabla) S = 0,
 \end{equation} 
 
 também usamos a equação de Poisson para determinar o potencial gravitacional
 
 \begin{equation}\label{eq8}
 	\nabla^2\phi = 4\pi G\varepsilon.
 \end{equation}

Através dessas equações hidrodinâmicas, podemos estudar o comportamento de pequenas pertubações nas funções desconhecidas: $ \varepsilon $,$ \textbf{V} $, $ p $ e $ \phi $.

\section*{Teoria de Jeans}

Assumimos que o universo é estático, homogêneo, isotrópico, não expansível e que a distribuição de energia na região estudada não varia com tempo e permaneça constante $\varepsilon (\mathbf{x},t) = \text{constante}$, mas para que a distribuição de energia permaneça constante, a matéria necessitaria de estar em repouso, o fato da força gravitacional ser proporcional ao gradiente do potencial, não permite que essa condição seja satisfeita, ou seja, a equação de Poisson não é satisfeita, logo o universo precisar de uma constante de cosmológica apropriada para se manter estático.

Com pequenas pertubações, temos

\begin{equation}\label{eq9}
	\varepsilon (\textbf{x},t) = \varepsilon_0 + \delta\varepsilon (\textbf{x},t),\,\, \textbf{V} (\textbf{x},t) = \textbf{V}_0 +\delta\textbf{V} (\textbf{x},t) 
\end{equation}

$$\phi (\textbf{x},t) = \phi_0 + \delta\phi (\textbf{x},t), S (\textbf{x},t)= S_0 + \delta S (\textbf{x},t)$$

onde cada variação $\delta\varepsilon \ll \varepsilon_0$ e assim por diante. A pressão é dada por

\begin{equation}\label{eq10}
	p (\textbf{x},t) = p( \varepsilon_0 + \delta\varepsilon (\textbf{x},t), S_0 + \delta S (\textbf{x},t) ) = p_0 +\delta p (\textbf{x},t) 
\end{equation}

ao realizar uma aproximação linear dessas pertubações, podemos escrever a pertubação na pressão em termos da pertubação da distribuição de energia e a entropia assim

\begin{equation}\label{eq11}
	\delta p = c_s^2\delta\varepsilon + \sigma\delta S
\end{equation}

onde $c^2_s \equiv \left(\dfrac{\partial p}{\partial\varepsilon}\right)_s$ é o quadrado da velocidade do som e $\sigma \equiv \left(\dfrac{\partial p}{\partial S}\right)_\varepsilon$. Para a matéria não relativística, a velocidade do som é muito menor que a velocidade da luz ($c_s \ll c  $).

Através dessas pertubações nas variáveis do sistema, conseguimos construir uma descrição clara com aproximações lineares do comportamento do sistema. Ao aplicar as pequenas pertubações em cada uma das definições acima, combinando as equações, conseguimos obter uma relação de cada uma das variações ao mantendo apenas os termos lineares as pertubações da seguinte forma 

\begin{equation}\label{eq12}
	\dfrac{\partial^2\delta\varepsilon}{\partial t^2} - c^2_s\nabla^2\delta\varepsilon - 4\pi G\varepsilon_0\delta\varepsilon = \sigma\nabla^2\delta S(\textbf{x}),
\end{equation}

$ \delta\varepsilon $ está contida em uma equação fechada, onde a entropia serve como uma fonte da pertubação.

\subsection*{Pertubações adiabáticas}

No caso de uma pertubação adiabática, onde não há troca de calor, ou seja, a pertubação na entropia é nula, e portanto, não depende das coordenadas espaciais, uma vez que a entropia é o único termo da eq.(12) depende exclusivamente das coordenadas espaciais, consequentemente, o termo a direita da equação eq.(12) se torna equivalente a 0, logo ao utilizar as transformações de Fourier e encontrar uma equação diferencial que relaciona a pertubação da distribuição de energia com as quantidades expressas em uma equação completamente dependente do tempo dada por 

\begin{equation}\label{eq13}
	\dfrac{\partial^2 (\delta\varepsilon_k )}{\partial t^2} +(k^2c^2_s - 4\pi G\varepsilon_0)\delta\varepsilon_k (t) = 0
\end{equation}

a solução é dada por 

\begin{equation}\label{eq14}
	\delta\varepsilon_k (t) \propto e^{\pm \omega (t) i}
\end{equation}

onde $\omega (t) = \sqrt{k^2c^2_s - 4\pi G \varepsilon_0}$ e o comportamento da pertubação adiabática depende exclusivamente do sinal do expoente.

Definindo o comprimento Jeans como

\begin{equation}\label{eq15}
	\lambda_J = \dfrac{2\pi}{k_J} = c_S \left(\dfrac{\pi}{G\varepsilon_0} \right)^{1/2},
\end{equation}

onde $\omega (k_J) = 0$, para  $\lambda < \lambda_J$, as soluções descrevem as ondas sonoras, proporcionais a $ \delta\varepsilon_k \propto \sin (\omega t + \mathbf{k}\mathbf{x} + \alpha) $ que propaga com velocidade de fase 

\begin{equation}\label{eq16}
	c_{fase} = \dfrac{\omega}{k}= c_s\sqrt{1 - \dfrac{k^2_J}{k}}.
\end{equation}

Ao analisar as soluções, no limite de $k \geq k_J$ ou em escalas muito pequenas ($\lambda \leq \lambda_J$), onde a contribuição da gravidade é insignificante comparado com a contribuição da pressão, consequentemente $c_{fase} \to c_s$.\\

Uma dessas soluções descrevem o comportamento exponencialmente rápido e não homogêneo, enquanto outras correspondem o modo decaimento, onde $k \to 0$, $|\omega | t \to \dfrac{t}{t_{gr}}$, onde $t_{gr} \equiv (4\pi G\varepsilon_0)^{-1/2}$. Onde $t_{gr}$ é interpretado como o tempo característico de colapso para uma região com densidade de energia inicial $\varepsilon_0$.

O comprimento Jeans $\lambda_J \sim c_s t_{gr} $ é o "comunicação de som" sobre a qual a pressão consegue reagir as mudanças da densidade de energia devido ao colapso gravitacional.


\subsection*{Pertubações de Vetor}
No caso de uma pertubação de vetor, onde as pertubações na distribuição de energia e na entropia são nulas, ou seja, $ \delta\varepsilon = 0 $ e $ \delta S = 0$, ao aplicar este resultado nas equações hidrodinâmicas de pertubações e a transformação de Fourier, concluímos que a velocidade da pertubação de onda plana, isto é, $\delta\mathbf{v} = \mathbf{w_k} e^{i\mathbf{k}\mathbf{x}}$ e que a velocidade é perpendicular ao vetor de onda $\mathbf{k}$, ou seja,

\begin{equation}\label{eq17}
	\mathbf{w_k} \cdot \mathbf{k} = 0.
\end{equation}

 As perturbações vetoriais descrevem o movimento de cisalhamento do meio que não perturbam a densidade de energia, pois existem duas direções perpendiculares independentes para $\mathbf{k}$.
 
 \subsection*{Pertubações de Entropia}
 
 Por fim, as pertubações de entropia, onde $ \delta S \neq 0$, ao analisar a eq. (12) completa, aplicando as transformações de Fourier obtemos a seguinte equação diferencial 
 
\begin{equation}\label{eq18}
 	\dfrac{\partial^2 (\delta\varepsilon_k )}{\partial t^2} +(k^2c^2_s - 4\pi G\varepsilon_0)\delta\varepsilon_k  = -\sigma k^2 \delta S_k
\end{equation} 
 
 onde a solução geral é a combinação da solução da homogênea e da solução particular, para o caso independente do tempo $  \dfrac{\partial^2 (\delta\varepsilon_k )}{\partial t^2}  = 0 $, a solução é dada por 
 
\begin{equation}\label{eq19}
	\delta\varepsilon_k  = \dfrac{ -\sigma k^2 \delta S_k}{k^2c^2_s - 4\pi G\varepsilon_0}
\end{equation}

é chamada de pertubação de entropia. Observe-se que $k \to \infty$ quando a contribuição da gravidade não é relevante. Neste caso, a contribuição para a pressão devido à falta de homogeneidade da densidade de energia é exatamente compensada pela contribuição correspondente as perturbações de entropia, logo, essas pertubações ocorrem apenas em fluidos constituídos de muitas componentes, por exemplo, um fluido que consiste em bárions e radiação.

\section*{Universo em Expansão}

Ao considerarmos a instabilidade gravitacional no universo em expansão, homogêneo e isotrópico, onde a distribuição de energia depende do tempo e as velocidades obedecem a lei de Hubble-Lemaître, temos

\begin{equation}\label{eq20}
	\varepsilon = \varepsilon_0 (t), \,\,\, \mathbf{V} = \mathbf{V}_0 = \mathbf{H} (t) \mathbf{x}.
\end{equation}
 
 substituímos esse resultado na eq. (3), obtemos
 
\begin{equation}\label{eq21}
	\dfrac{\partial \varepsilon_0}{\partial t}  + 3 \mathbf{H}\varepsilon_0 = 0 
\end{equation} 

ao agruparmos a divergência equação de Euler com a equação de Poisson, chegamos na equação de Friedmann

\begin{equation}\label{eq22}
	\dot{\mathbf{H}} + \mathbf{H}^2 = - \dfrac{4\pi G}{3}\varepsilon_0,
\end{equation}

onde as pertubações são dadas por

\begin{equation}\label{eq23}
	\varepsilon = \varepsilon_0 + \delta\varepsilon_0 (\mathbf{x},t),\,\, \mathbf{V} = \mathbf{V}_0 + \delta\mathbf{v} , \phi= \phi_0 + \delta\phi\,\,\, 
\end{equation}

$$p = p_0 + \delta p= p_0 + c_2^2\delta\varepsilon ,$$

ao ignorar a pertubação de entropia. Podemos relacionar novamente as equações anteriores de hidrodinâmica com os termos lineares as pertubações e encontrarmos as equações hidrodinâmicas para um universo em expansão.

A velocidade de Hubble-Lemaître depende exclusivamente da posição $ \mathbf{x} $, portanto as transformações de Fourier não desacoplam as coordenadas eulerianas $ \mathbf{x} $ em equações diferenciais ordinárias dependentes exclusivamente do tempo. Logo, é mais conveniente utilizar as coordenadas Lagrangianas $ \mathbf{q} $ com  

\begin{equation}\label{eq24}
	\mathbf{x} = a(t)\mathbf{q},
\end{equation}

relacionando as operações diferenciais para uma função genérica 

\begin{equation}\label{eq25}
	\left(\dfrac{\partial}{\partial t}\right)_\mathbf{x} = \left( \dfrac{\partial}{\partial t} \right)_\mathbf{q} - (\mathbf{V}_0 \cdot \nabla_\mathbf{x}),
\end{equation}
 
as devidas espaciais estão relacionadas de forma mais simples

\begin{equation}\label{eq26}
	\nabla_\mathbf{x} = \dfrac{1}{a}\nabla_\mathbf{q},
\end{equation}

 introduzindo a amplitude fracionária de densidade de energia $ \delta \equiv \frac{\delta\varepsilon}{\varepsilon_0} $, podemos reescrever as equações hidrodinâmicas para um universo em expansão, a equação que descreve a conservação do fluxo de massa assume a forma 
 
 \begin{equation}\label{eq27}
 	\left( \dfrac{\partial \delta}{\partial t} \right) + \dfrac{1}{a}(\nabla\cdot\delta\mathbf{v}) = 0,
 \end{equation}
 
a equação de Euler é dada por

\begin{equation}\label{eq28}
	\left( \dfrac{\partial \delta\mathbf{v}}{\partial t} \right) +H\delta\mathbf{v}+\dfrac{c^2_s}{a}\nabla\delta + \dfrac{1}{a}\nabla\delta\phi =0
\end{equation}

e por fim, a equação de Poisson

\begin{equation}\label{eq29}
	\nabla^2\delta\phi = 4\pi Ga^2\varepsilon_0\delta
\end{equation}
 
e agrupando as equações relacionando seus resultados em termos lineares as pequenas pertubações, obtemos a equação que descreve a instabilidade gravitacional no universo em expansão que dado por
 
\begin{equation}\label{eq30}
 	\ddot{\delta} + 2\mathbf{H}\dot{\delta} - \dfrac{c^2_s}{a^2}\nabla^2\delta - 4\pi G\varepsilon_0\delta = 0.
\end{equation}

\subsection*{Pertubações adiabáticas}

No caso de pertubações adiabáticas no universo em expansão, utilizando a descrição da amplitude fracionária de densidade de energia no espaço de Fourier, obtemos a seguinte equação diferencial 

\begin{equation}\label{eq31}
	\ddot{\delta_\mathbf{k}} + 2\mathbf{H}\dot{\delta_\mathbf{k}} + \left( \dfrac{c^2_s k^2}{a^2} -4\pi G\varepsilon_0\right)\delta_\mathbf{k} = 0,
\end{equation} 

onde o comportamento de cada pertubação depende especialmente do tamanho espacial e o comprimento de Jeans é dado por

\begin{equation}\label{eq32}
	\lambda^{ph}_J = \dfrac{2\pi a }{k_J} = c_s\sqrt{\dfrac{\pi}{G\varepsilon_0}}
\end{equation}

onde $\lambda^{ph}$ é o comprimento de onda físico, relacionando
ao comprimento de onda comovente $\lambda = \dfrac{2\pi}{k}$ com $\lambda^{ph} = a\lambda$. Em um universo plano dominado pela matéria $\varepsilon_0 = (6\pi G t^2)^{-1}$ e, portanto,

\begin{equation}\label{eq33}
	\lambda_J^{ph} \sim c_s t,
\end{equation}

isto é, o comprimento do Jeans está na ordem do horizonte de som. Às vezes, ao  invés de usar o comprimento de Jeans, usa-se a massa de Jeans, definida como $M_J \equiv\varepsilon_0 (\lambda_J^{ph})^3$. Se $ c_s $ mudar diabaticamente, a solução particular da eq. (31) é dada por

\begin{equation}\label{eq34}
	\delta_\mathbf{k} \propto \dfrac{1}{\sqrt{c_s a}} \,exp\left( \pm k \int \dfrac{c_s dt}{a}\right)
\end{equation}

Em escalas muito maiores do que a escala de Jeans, a gravidade possui a maior contribuição na evolução da pertubação domina e negligencia o termo dependente de k em eq.(31). Então, uma das soluções é simplesmente e proporcional a constante de Hubble. Entretanto, ao substituir essa solução  $\delta_d = H (t)$ na eq.(31) e assumindo $c^2_s k^2 = 0$, descobre-se que a equação resultante coincide com o tempo derivada da equação de Friedmann eq.(22). Observe que $\delta_d = H (t)$ é o decaimento
solução da equação de perturbação ($ H $ diminui com o tempo) em um dominado pela matéria universo com curvatura arbitrária.

Quando utilizamos as propriedades da derivada do Wronskiano para encontrar soluções descobrimos que as soluções são proporcionais termos lineares a potência do tempo, portanto, vemos que em um universo em expansão, a instabilidade gravitacional é muito menor eficiente e a amplitude da perturbação aumenta apenas com a potência do tempo.

\subsection*{Pertubações de vetor}

No caso de pertubações de vetor em um universo em expansão, ou seja, substituindo $ \delta = 0$ nas equações hidrodinâmicas, obtemos

\begin{equation}\label{eq35}
	\nabla\delta\mathbf{v} = 0 , \quad \dfrac{\partial\delta\mathbf{v}}{\partial t } + H\delta\mathbf{v} = 0
\end{equation}

A primeira equação segue que para uma perturbação de onda plana, $\delta\mathbf{v} \propto \delta\mathbf{v}_\mathbf{k} (t)e^{(i\mathbf{kq})}$ e a velocidade peculiar $\delta\mathbf{v}$ é perpendicular ao número de onda $\mathbf{k}$. A segunda equação torna-se

\begin{equation}\label{eq36}
	\delta\dot{\mathbf{v}}_\mathbf{k}+ \dfrac{\dot{a}}{a}\delta\mathbf{v}_\mathbf{k}=0,
\end{equation}

e tem a solução $\delta\mathbf{v_k} \propto \frac{1}{a}$. Assim, as perturbações do vetor decaem conforme o
o universo se expande e podem ter amplitudes significativas no presente
apenas se suas amplitudes iniciais fossem tão grandes que estragassem completamente a isotropia do universo primordial. Em um universo inflacionário, não há espaço para grandes perturbações vetoriais primordiais e não desempenham qualquer papel na formação de
a estrutura em grande escala no universo. Perturbações de vetor, no entanto, podem ser
gerado tardiamente, após a estrutura não linear ter sido formada e pode explicar a rotação das galáxias por exemplo.